\documentclass[uplatex,a4paper,10pt]{jsarticle}
\usepackage[dvipdfmx]{graphicx}
\usepackage{geometry}
\geometry{margin=20mm}
\usepackage{titlesec}
\usepackage{array}
\usepackage{enumitem}
\usepackage{hyperref}
\hypersetup{colorlinks=true,linkcolor=black,urlcolor=black}

% セクション見出し(太字・大きめ)
\titleformat{\section}{\large\bfseries}{\thesection}{1em}{}
\titlespacing{\section}{0pt}{8pt}{4pt}

% 行間と項目の詰め
\setlist[itemize]{leftmargin=*, itemsep=2pt, topsep=2pt}
\setlength{\parskip}{2pt}
\setlength{\parindent}{0pt}

% 左:太字タイトル 右:日付  下:本文 というCV行
\newcommand{\cvitem}[3]{%
  \begin{tabular*}{\textwidth}{@{}p{0.72\textwidth}@{\extracolsep{\fill}}p{0.25\textwidth}@{}}
    \textbf{#1} & \raggedleft #2 \\
  \end{tabular*}\\[-2pt]
  #3\par\vspace{4pt}
}

% ヘッダー(氏名+連絡先)
\newcommand{\cvheader}[3]{%
  {\LARGE \textbf{#1}}\\[4pt]
  #2\\
  #3
}

\begin{document}

% ==== ヘッダー ====
\cvheader{Takahiro Matsunaga}
{7-3-1 Hongo, Bunkyo-ku, Tokyo 113-8656, Japan}
{(+81) xx-xxxx-xxxx \ \ $\diamond$ \ \ \href{mailto:matsunaga@bin.t.u-tokyo.ac.jp}{matsunaga@bin.t.u-tokyo.ac.jp} \ \ $\diamond$ \ \ \href{https://scholar.google.com/citations?user=fGWoyocAAAAJ}{Google Scholar}}

\vspace{8pt}

% ==== EXPERIENCE ====
\section*{EXPERIENCE}
\cvitem{Research Assistant, The University of Tokyo}{2025.04--Present}
{Behavior in Networks (BinN) Studies Unit. Advisor: Prof. Eiji Hato}

\cvitem{Visiting Research Student, XYZ Institute}{2024.04--2025.03}
{Multi-agent/sequence modeling for transportation networks.}

% ==== EDUCATION ====
\section*{EDUCATION}
\cvitem{The University of Tokyo}{2025--}
{M.S./Ph.D. in Civil Engineering (Transportation). Thesis: \emph{[Title]}}

\cvitem{[Your Previous Univ.]}{2019--2023}
{B.S. in [Major].}

% ==== PUBLICATIONS(査読誌) ====
\section*{PUBLICATIONS (Peer-Reviewed Journals)}
\begin{itemize}
  \item Matsunaga, T., Hato, E. (2024).
  BLE観測の不確実性を考慮した屋内3次元経路選択モデルの推定.
  \emph{都市計画論文集}, 59(3): 1683--1690.
\end{itemize}

% ==== CONFERENCES ====
\section*{PRESENTATIONS (International Conferences)}
\begin{itemize}
  \item Matsunaga, T., Hato, E. (2025).
  Interaction of 3D pedestrian flow in a congested railway station: Structural estimation based on Mean Field Game theory.
  The 12th PED, Prague, Czech Republic (accepted).
\end{itemize}

\section*{PRESENTATIONS (Domestic Conferences)}
\begin{itemize}
  \item 松永 隆宏, 羽藤 英二 (2025).
  複雑なday-to-day活動動態学習に基づく首都圏アクティビティ生成モデル.
  交通工学研究発表会(口頭発表予定).
\end{itemize}

% ==== SERVICE ====
\section*{SERVICE}
\begin{itemize}
  \item Reviewer: IEEE ITSC, TRB, etc.
\end{itemize}

% ==== HONORS & AWARDS ====
\section*{HONORS \& AWARDS}
\begin{itemize}
  \item Best Graduation Thesis, Dept. of Civil Engineering, The University of Tokyo (2023).
  \item Outstanding Master Thesis, Dept. of Civil Engineering, The University of Tokyo (2025).
\end{itemize}

% ==== LEADERSHIP ====
\section*{LEADERSHIP}
\begin{itemize}
  \item [Activity / Role], Organization (Year).
\end{itemize}

\end{document}
